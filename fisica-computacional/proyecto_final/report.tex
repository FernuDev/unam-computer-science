\documentclass[12pt,letterpaper]{article}

% Paquetes esenciales
\usepackage[utf8]{inputenc}
\usepackage[spanish]{babel}
\usepackage{amsmath,amssymb,amsthm}
\usepackage{mathtools}
\usepackage{physics}
\usepackage{geometry}
\usepackage{graphicx}
\usepackage{hyperref}
\usepackage{float}
\usepackage{enumitem}
\usepackage{tikz}
\usepackage{pgfplots}
\usepackage{listings}
\usepackage{xcolor}
\pgfplotsset{compat=1.18}

% Configuración de listings para Python
\lstset{
    language=Python,
    basicstyle=\ttfamily\small,
    keywordstyle=\color{blue}\bfseries,
    commentstyle=\color{green!60!black},
    stringstyle=\color{red},
    showstringspaces=false,
    numbers=left,
    numberstyle=\tiny\color{gray},
    frame=single,
    breaklines=true,
    captionpos=b
}

% Configuración de página
\geometry{
    left=2.5cm,
    right=2.5cm,
    top=3cm,
    bottom=3cm
}

% Estilos de teoremas
\theoremstyle{definition}
\newtheorem{definition}{Definición}[section]
\newtheorem{theorem}{Teorema}[section]
\newtheorem{lemma}[theorem]{Lema}
\newtheorem{proposition}[theorem]{Proposición}
\newtheorem{corollary}[theorem]{Corolario}
\theoremstyle{remark}
\newtheorem{remark}{Observación}[section]
\newtheorem{example}{Ejemplo}[section]

% Operadores matemáticos personalizados
\DeclareMathOperator{\sign}{sign}
\DeclareMathOperator{\argmin}{argmin}

% Información del documento
\title{Análisis Matemático del Sistema de Intercepción de Misiles}
\author{Universidad Nacional Autónoma de México}
\date{\today}

\begin{document}

% Portada personalizada
\begin{titlepage}
    \centering
    
    % Logo y encabezado
    \vspace*{1cm}
    {\Large\textbf{UNIVERSIDAD NACIONAL AUTÓNOMA DE MÉXICO}}\\
    \vspace{0.3cm}
    {\large\textsc{Facultad de Ciencias}}\\
    \vspace{0.2cm}
    {\large Física Computacional}\\
    
    \vspace{2cm}
    
    % Línea decorativa
    \rule{\textwidth}{0.5pt}
    \vspace{0.3cm}
    
    % Título principal
    {\Huge\textbf{Análisis Matemático Riguroso del}}\\
    \vspace{0.4cm}
    {\Huge\textbf{Sistema de Ecuaciones Diferenciales}}\\
    \vspace{0.4cm}
    {\Huge\textbf{para la Simulación de}}\\
    \vspace{0.4cm}
    {\Huge\textbf{Intercepción de Misiles en 3D}}\\
    
    \vspace{0.3cm}
    \rule{\textwidth}{0.5pt}
    
    \vspace{1.5cm}
    
    % Subtítulo
    {\LARGE\textit{Sistema de Navegación Proporcional y}}\\
    \vspace{0.3cm}
    {\LARGE\textit{Métodos de Integración Numérica}}\\
    
    \vspace{2cm}
    
    % Información del documento
    \begin{tabular}{rl}
        \textbf{Proyecto:} & Proyecto Final \\
        \textbf{Curso:} & Física Computacional \\
        \textbf{Institución:} & UNAM -- Facultad de Ciencias \\
        \textbf{Fecha:} & \today \\
    \end{tabular}
    
    \vfill
    
    % Descripción breve
    \begin{minipage}{0.8\textwidth}
        \centering
        \small
        \textit{Análisis exhaustivo del sistema acoplado de 12 ecuaciones diferenciales ordinarias no lineales que gobierna la dinámica de intercepción tridimensional entre un misil táctico y una aeronave maniobrable, incluyendo derivaciones rigurosas de la ley de navegación proporcional, análisis de estabilidad mediante teoría de Lyapunov, e implementación computacional mediante métodos Runge-Kutta de cuarto orden.}
    \end{minipage}
    
    \vspace{1cm}
    
    % Pie de página
    {\small Ciudad de México, México}
    
\end{titlepage}

% Página en blanco después de la portada
\cleardoublepage
\tableofcontents
\newpage

%==============================================================================
\section{Introducción}
%==============================================================================

La simulación computacional de sistemas de guiado de misiles constituye un problema fundamental en la teoría de control óptimo, dinámica de vuelo y ecuaciones diferenciales no lineales. Este documento presenta un análisis matemático exhaustivo y riguroso del sistema de ecuaciones diferenciales ordinarias (EDOs) que gobierna la dinámica de intercepción tridimensional entre un misil táctico y una aeronave maniobrable, abordando el problema desde los primeros principios de la física matemática.

\subsection{Contexto Físico del Problema}

El problema de intercepción de misiles se enmarca dentro de la teoría de juegos diferenciales, específicamente en la clase de problemas de persecución-evasión (pursuit-evasion games). En este contexto, dos agentes en un espacio métrico compiten: el perseguidor (misil) busca minimizar la distancia relativa, mientras que el evadido (objetivo) busca maximizarla o al menos retardar la intercepción.

\subsection{Planteamiento del Problema}

Consideremos dos objetos puntuales en el espacio tridimensional $\mathbb{R}^3$:
\begin{itemize}[leftmargin=*]
    \item Un \textbf{objetivo} (aeronave), denotado por $T$ (Target)
    \item Un \textbf{interceptor} (misil), denotado por $M$ (Missile)
\end{itemize}

\begin{definition}[Problema de Intercepción]
Dado un estado inicial $\mathbf{S}(0) = \mathbf{S}_0 \in \mathbb{R}^{12}$, encontrar una ley de control admisible $\mathbf{a}_M: [0,\infty) \times \mathbb{R}^{12} \to \mathbb{R}^3$ tal que:
\begin{equation}
    \exists t_f < \infty : \|\mathbf{r}_T(t_f) - \mathbf{r}_M(t_f)\| \leq \epsilon
\end{equation}
donde $\epsilon > 0$ es una tolerancia prescrita (radio letal del misil).
\end{definition}

Este es un problema de \textit{persecución diferencial} (differential pursuit) en el cual el perseguidor (misil) debe interceptar a un objetivo evasivo que ejecuta maniobras complejas, sujeto a las siguientes restricciones físicas:

\begin{align}
    \|\mathbf{a}_M(t)\| &\leq a_{max} \quad \text{(límite estructural de aceleración)} \\
    \|\mathbf{v}_M(t)\| &\leq v_{max} \quad \text{(límite de velocidad por arrastre aerodinámico)} \\
    \|\mathbf{a}_T(t)\| &\leq a_T^{max} \quad \text{(capacidad de maniobra del objetivo)}
\end{align}

%==============================================================================
\section{Espacio de Estados y Formulación del Sistema Dinámico}
%==============================================================================

\subsection{Vector de Estado}

El sistema completo se describe mediante un vector de estado de dimensión 12:

\begin{definition}[Vector de Estado del Sistema]
    El estado completo del sistema en un instante $t$ está dado por:
    \begin{equation}
        \mathbf{S}(t) = \begin{bmatrix}
            \mathbf{r}_T(t) \\
            \mathbf{v}_T(t) \\
            \mathbf{r}_M(t) \\
            \mathbf{v}_M(t)
        \end{bmatrix} \in \mathbb{R}^{12}
    \end{equation}
    donde:
    \begin{align}
        \mathbf{r}_T(t) &= [x_T(t), \, y_T(t), \, z_T(t)]^T \in \mathbb{R}^3 \quad \text{(posición del objetivo)} \\
        \mathbf{v}_T(t) &= [\dot{x}_T(t), \, \dot{y}_T(t), \, \dot{z}_T(t)]^T \in \mathbb{R}^3 \quad \text{(velocidad del objetivo)} \\
        \mathbf{r}_M(t) &= [x_M(t), \, y_M(t), \, z_M(t)]^T \in \mathbb{R}^3 \quad \text{(posición del misil)} \\
        \mathbf{v}_M(t) &= [\dot{x}_M(t), \, \dot{y}_M(t), \, \dot{z}_M(t)]^T \in \mathbb{R}^3 \quad \text{(velocidad del misil)}
    \end{align}
\end{definition}

\subsection{Sistema de Ecuaciones Diferenciales Ordinarias}

El sistema se modela como un problema de valor inicial (PVI) para un sistema autónomo de EDOs de primer orden:

\begin{equation}
    \frac{d\mathbf{S}}{dt} = \mathbf{F}(t, \mathbf{S})
    \label{eq:system_ode}
\end{equation}

con condición inicial:
\begin{equation}
    \mathbf{S}(0) = \mathbf{S}_0
\end{equation}

\begin{theorem}[Sistema de 12 EDOs Acopladas]
    El sistema \eqref{eq:system_ode} se descompone en las siguientes ecuaciones:
    
    \textbf{Parte Cinemática (6 ecuaciones):}
    \begin{align}
        \frac{d\mathbf{r}_T}{dt} &= \mathbf{v}_T \label{eq:kinematics_1} \\
        \frac{d\mathbf{r}_M}{dt} &= \mathbf{v}_M \label{eq:kinematics_2}
    \end{align}
    
    \textbf{Parte Dinámica (6 ecuaciones):}
    \begin{align}
        \frac{d\mathbf{v}_T}{dt} &= \mathbf{a}_T(t, \mathbf{r}_T, \mathbf{v}_T) \label{eq:dynamics_target} \\
        \frac{d\mathbf{v}_M}{dt} &= \mathbf{a}_M(t, \mathbf{r}_M, \mathbf{v}_M, \mathbf{r}_T, \mathbf{v}_T) \label{eq:dynamics_missile}
    \end{align}
\end{theorem}

En forma explícita, el sistema completo es:
\begin{equation}
    \frac{d}{dt}\begin{bmatrix}
        x_T \\ y_T \\ z_T \\ \dot{x}_T \\ \dot{y}_T \\ \dot{z}_T \\ x_M \\ y_M \\ z_M \\ \dot{x}_M \\ \dot{y}_M \\ \dot{z}_M
    \end{bmatrix} = \begin{bmatrix}
        \dot{x}_T \\ \dot{y}_T \\ \dot{z}_T \\ a_{T,x} \\ a_{T,y} \\ a_{T,z} \\ \dot{x}_M \\ \dot{y}_M \\ \dot{z}_M \\ a_{M,x} \\ a_{M,y} \\ a_{M,z}
    \end{bmatrix}
    \label{eq:full_system}
\end{equation}

\subsection{Estructura Matricial del Sistema}

El sistema \eqref{eq:full_system} puede expresarse en forma matricial compacta que revela la estructura de bloques inherente.

\subsubsection{Forma Canónica de Primer Orden}

Definiendo las particiones del vector de estado:
\begin{equation}
    \mathbf{q} = \begin{bmatrix} \mathbf{r}_T \\ \mathbf{r}_M \end{bmatrix} \in \mathbb{R}^6, \quad
    \mathbf{p} = \begin{bmatrix} \mathbf{v}_T \\ \mathbf{v}_M \end{bmatrix} \in \mathbb{R}^6
\end{equation}

El sistema se puede escribir como:
\begin{equation}
    \begin{bmatrix}
        \dot{\mathbf{q}} \\
        \dot{\mathbf{p}}
    \end{bmatrix} = \begin{bmatrix}
        \mathbf{p} \\
        \mathbf{A}(t, \mathbf{q}, \mathbf{p})
    \end{bmatrix}
    \label{eq:canonical_form}
\end{equation}

donde $\mathbf{A}(t, \mathbf{q}, \mathbf{p}) \in \mathbb{R}^6$ es el vector de aceleraciones:
\begin{equation}
    \mathbf{A} = \begin{bmatrix}
        \mathbf{a}_T(t, \mathbf{r}_T, \mathbf{v}_T) \\
        \mathbf{a}_M(t, \mathbf{r}_M, \mathbf{v}_M, \mathbf{r}_T, \mathbf{v}_T)
    \end{bmatrix}
\end{equation}

\subsubsection{Representación en Forma de Hamilton}

Aunque el sistema no es Hamiltoniano (debido a la disipación y control activo), podemos expresarlo en una forma análoga:
\begin{align}
    \frac{d\mathbf{q}}{dt} &= \mathbf{p} \label{eq:hamilton_q} \\
    \frac{d\mathbf{p}}{dt} &= \mathbf{F}(t, \mathbf{q}, \mathbf{p}) \label{eq:hamilton_p}
\end{align}

Esta separación exhibe la estructura geométrica del espacio de fase $\mathcal{M} = \mathbb{R}^6 \times \mathbb{R}^6$.

\subsection{Análisis del Acoplamiento No Lineal}

El sistema posee un acoplamiento no lineal fuerte entre las variables del objetivo y del misil a través de las leyes de guiado.

\subsubsection{Diagrama de Dependencias}

Las ecuaciones exhiben la siguiente estructura de dependencias:

\begin{enumerate}
    \item $\dot{\mathbf{r}}_T = \mathbf{v}_T$ (cinemática desacoplada)
    \item $\dot{\mathbf{r}}_M = \mathbf{v}_M$ (cinemática desacoplada)
    \item $\dot{\mathbf{v}}_T = \mathbf{a}_T(t, \mathbf{r}_T, \mathbf{v}_T)$ (acoplamiento débil: solo depende del objetivo)
    \item $\dot{\mathbf{v}}_M = \mathbf{a}_M(t, \mathbf{r}_M, \mathbf{v}_M, \mathbf{r}_T, \mathbf{v}_T)$ (acoplamiento fuerte: depende de ambos agentes)
\end{enumerate}

\begin{definition}[Matriz de Dependencias]
Definimos la matriz de dependencias funcionales $\mathcal{D} \in \{0,1\}^{4 \times 4}$ donde $\mathcal{D}_{ij} = 1$ si y solo si el bloque de ecuaciones $i$ depende del bloque de variables $j$:
\begin{equation}
    \mathcal{D} = \begin{array}{c|cccc}
         & \mathbf{r}_T & \mathbf{v}_T & \mathbf{r}_M & \mathbf{v}_M \\ \hline
        \dot{\mathbf{r}}_T & 0 & 1 & 0 & 0 \\
        \dot{\mathbf{v}}_T & 1 & 1 & 0 & 0 \\
        \dot{\mathbf{r}}_M & 0 & 0 & 0 & 1 \\
        \dot{\mathbf{v}}_M & 1 & 1 & 1 & 1
    \end{array}
\end{equation}
\end{definition}

\subsubsection{Caracterización del Acoplamiento}

El acoplamiento entre misil y objetivo se manifiesta a través de la función de guiado $\mathbf{g}$:
\begin{equation}
    \mathbf{a}_M = \mathbf{g}(\mathbf{r}_M, \mathbf{v}_M, \mathbf{r}_T, \mathbf{v}_T) = \mathbf{g}(\mathbf{r}_{rel}, \mathbf{v}_{rel})
\end{equation}

Esta función tiene las siguientes propiedades:

\begin{proposition}[Homogeneidad de la Ley PN]
La ley de navegación proporcional pura es homogénea de grado 0 en las posiciones:
\begin{equation}
    \mathbf{g}(\lambda\mathbf{r}_{rel}, \mathbf{v}_{rel}) = \mathbf{g}(\mathbf{r}_{rel}, \mathbf{v}_{rel}), \quad \forall \lambda > 0
\end{equation}
esto es, la ley PN depende solo de la dirección relativa, no de la magnitud de la separación.
\end{proposition}

\begin{proof}
Desde la ecuación \eqref{eq:pn_law_main}:
\begin{align}
    \mathbf{a}_{PN} &= N V_c (\boldsymbol{\Omega} \times \hat{\mathbf{r}}_{LOS}) \\
    &= N \cdot \left(-\frac{\mathbf{r}_{rel} \cdot \mathbf{v}_{rel}}{R}\right) \cdot \frac{(\mathbf{r}_{rel} \times \mathbf{v}_{rel})}{R^2} \times \frac{\mathbf{r}_{rel}}{R}
\end{align}

Sustituyendo $\mathbf{r}_{rel} \to \lambda\mathbf{r}_{rel}$:
\begin{align}
    V_c &\to -\frac{\lambda\mathbf{r}_{rel} \cdot \mathbf{v}_{rel}}{\lambda R} = -\frac{\mathbf{r}_{rel} \cdot \mathbf{v}_{rel}}{R} \\
    \boldsymbol{\Omega} &\to \frac{\lambda\mathbf{r}_{rel} \times \mathbf{v}_{rel}}{(\lambda R)^2} = \frac{\mathbf{r}_{rel} \times \mathbf{v}_{rel}}{\lambda R^2} \\
    \hat{\mathbf{r}}_{LOS} &\to \frac{\lambda\mathbf{r}_{rel}}{\lambda R} = \frac{\mathbf{r}_{rel}}{R}
\end{align}

Por lo tanto, la combinación $V_c \cdot \boldsymbol{\Omega}$ cancela el factor $\lambda$ y $\mathbf{a}_{PN}$ permanece invariante.
\end{proof}

\subsection{Análisis del Jacobiano}

Para el análisis de estabilidad local, calculamos la matriz Jacobiana del sistema.

\subsubsection{Derivación de la Matriz Jacobiana}

La matriz Jacobiana $\mathbf{J} \in \mathbb{R}^{12 \times 12}$ se define como:
\begin{equation}
    \mathbf{J}_{ij} = \frac{\partial F_i}{\partial S_j}
\end{equation}

Dada la estructura del sistema, el Jacobiano tiene forma de bloques:
\begin{equation}
    \mathbf{J} = \begin{bmatrix}
        \mathbf{0}_{3\times3} & \mathbf{I}_3 & \mathbf{0}_{3\times3} & \mathbf{0}_{3\times3} \\
        \frac{\partial \mathbf{a}_T}{\partial \mathbf{r}_T} & \frac{\partial \mathbf{a}_T}{\partial \mathbf{v}_T} & \mathbf{0}_{3\times3} & \mathbf{0}_{3\times3} \\
        \mathbf{0}_{3\times3} & \mathbf{0}_{3\times3} & \mathbf{0}_{3\times3} & \mathbf{I}_3 \\
        \frac{\partial \mathbf{a}_M}{\partial \mathbf{r}_T} & \frac{\partial \mathbf{a}_M}{\partial \mathbf{v}_T} & \frac{\partial \mathbf{a}_M}{\partial \mathbf{r}_M} & \frac{\partial \mathbf{a}_M}{\partial \mathbf{v}_M}
    \end{bmatrix}
    \label{eq:jacobian_structure}
\end{equation}

\subsubsection{Cálculo de las Derivadas Parciales}

\textbf{Para el objetivo:}
\begin{align}
    \frac{\partial \mathbf{a}_T}{\partial \mathbf{r}_T} &= \mathbf{0}_{3\times3} \quad \text{(maniobra independiente de posición)} \\
    \frac{\partial \mathbf{a}_T}{\partial \mathbf{v}_T} &= -\frac{1}{\tau_{resp}}\mathbf{I}_3
\end{align}

\textbf{Para el misil (PN pura):}

Las derivadas de $\mathbf{a}_{PN}$ son más complejas. Comenzamos con:
\begin{equation}
    \mathbf{a}_{PN} = N V_c (\boldsymbol{\Omega} \times \hat{\mathbf{r}}_{LOS})
\end{equation}

Utilizando la regla del producto:
\begin{align}
    \frac{\partial \mathbf{a}_{PN}}{\partial \mathbf{r}_T} &= N\left[\frac{\partial V_c}{\partial \mathbf{r}_T}(\boldsymbol{\Omega} \times \hat{\mathbf{r}}_{LOS}) + V_c\frac{\partial}{\partial \mathbf{r}_T}(\boldsymbol{\Omega} \times \hat{\mathbf{r}}_{LOS})\right]
\end{align}

Cada término requiere la aplicación de la regla de la cadena a través de $\mathbf{r}_{rel}$ y $R$.

\begin{theorem}[Gradiente de Velocidad de Cierre]
\begin{equation}
    \nabla_{\mathbf{r}_T} V_c = \frac{1}{R}\left[\mathbf{v}_{rel} - \hat{\mathbf{r}}_{LOS}(\hat{\mathbf{r}}_{LOS} \cdot \mathbf{v}_{rel})\right] = \frac{1}{R}\mathbf{v}_{rel}^{\perp}
\end{equation}
\end{theorem}

\begin{proof}
Derivando $V_c = -\frac{\mathbf{r}_{rel} \cdot \mathbf{v}_{rel}}{R}$ respecto a $\mathbf{r}_T$:
\begin{align}
    \nabla_{\mathbf{r}_T} V_c &= -\nabla_{\mathbf{r}_T}\left(\frac{\mathbf{r}_{rel} \cdot \mathbf{v}_{rel}}{R}\right) \\
    &= -\left[\frac{\nabla_{\mathbf{r}_T}(\mathbf{r}_{rel} \cdot \mathbf{v}_{rel})}{R} - \frac{(\mathbf{r}_{rel} \cdot \mathbf{v}_{rel})\nabla_{\mathbf{r}_T} R}{R^2}\right]
\end{align}

Con $\nabla_{\mathbf{r}_T}(\mathbf{r}_{rel} \cdot \mathbf{v}_{rel}) = \mathbf{v}_{rel}$ y $\nabla_{\mathbf{r}_T} R = \hat{\mathbf{r}}_{LOS}$:
\begin{align}
    \nabla_{\mathbf{r}_T} V_c &= -\frac{\mathbf{v}_{rel}}{R} + \frac{(\mathbf{r}_{rel} \cdot \mathbf{v}_{rel})\hat{\mathbf{r}}_{LOS}}{R^2} \\
    &= \frac{1}{R}\left[\hat{\mathbf{r}}_{LOS}(\hat{\mathbf{r}}_{LOS} \cdot \mathbf{v}_{rel}) - \mathbf{v}_{rel}\right] \\
    &= -\frac{1}{R}\mathbf{v}_{rel}^{\perp}
\end{align}
\end{proof}

\subsection{Propiedades del Sistema Dinámico}

\subsubsection{Espacio de Fase y Flujo}

El sistema define un flujo $\phi_t: \mathcal{M} \to \mathcal{M}$ en el espacio de fase 12-dimensional:
\begin{equation}
    \phi_t(\mathbf{S}_0) = \mathbf{S}(t; \mathbf{S}_0)
\end{equation}

que satisface:
\begin{align}
    \phi_0 &= \text{id}_{\mathcal{M}} \quad \text{(identidad)} \\
    \phi_{t+s} &= \phi_t \circ \phi_s \quad \text{(propiedad de semigrupo)}
\end{align}

\subsubsection{Invariantes y Conservación}

Aunque el sistema no es conservativo (debido al control activo), podemos identificar ciertas cantidades que evolucionan de manera predecible.

\begin{proposition}[Evolución del Momento Angular Relativo]
El momento angular relativo $\mathbf{L}_{rel} = \mathbf{r}_{rel} \times \mathbf{v}_{rel}$ satisface:
\begin{equation}
    \frac{d\mathbf{L}_{rel}}{dt} = \mathbf{r}_{rel} \times (\mathbf{a}_T - \mathbf{a}_M)
\end{equation}
\end{proposition}

\begin{proof}
\begin{align}
    \frac{d\mathbf{L}_{rel}}{dt} &= \frac{d}{dt}(\mathbf{r}_{rel} \times \mathbf{v}_{rel}) \\
    &= \dot{\mathbf{r}}_{rel} \times \mathbf{v}_{rel} + \mathbf{r}_{rel} \times \dot{\mathbf{v}}_{rel} \\
    &= \mathbf{v}_{rel} \times \mathbf{v}_{rel} + \mathbf{r}_{rel} \times (\mathbf{a}_T - \mathbf{a}_M) \\
    &= \mathbf{r}_{rel} \times (\mathbf{a}_T - \mathbf{a}_M)
\end{align}
donde usamos $\mathbf{v}_{rel} \times \mathbf{v}_{rel} = \mathbf{0}$.
\end{proof}

Esta ecuación muestra que el cambio en momento angular relativo está directamente relacionado con el torque aplicado por las aceleraciones.

%==============================================================================
\section{Cinemática Vectorial Relativa}
%==============================================================================

\subsection{Geometría de la Intercepción}

Definimos las siguientes cantidades vectoriales fundamentales:

\begin{definition}[Cinemática Relativa]
    \begin{align}
        \mathbf{r}_{rel}(t) &= \mathbf{r}_T(t) - \mathbf{r}_M(t) \quad \text{(vector de posición relativa)} \label{eq:r_rel} \\
        \mathbf{v}_{rel}(t) &= \mathbf{v}_T(t) - \mathbf{v}_M(t) \quad \text{(vector de velocidad relativa)} \label{eq:v_rel} \\
        R(t) &= \|\mathbf{r}_{rel}(t)\| = \sqrt{\mathbf{r}_{rel} \cdot \mathbf{r}_{rel}} \quad \text{(distancia)} \label{eq:distance}
    \end{align}
\end{definition}

\subsection{Línea de Visión (LOS)}

\begin{definition}[Vector Unitario de Línea de Visión]
    El vector unitario que apunta del misil al objetivo se define como:
    \begin{equation}
        \hat{\mathbf{r}}_{LOS}(t) = \frac{\mathbf{r}_{rel}(t)}{R(t)} = \frac{\mathbf{r}_T - \mathbf{r}_M}{\|\mathbf{r}_T - \mathbf{r}_M\|}
        \label{eq:los_unit}
    \end{equation}
\end{definition}

\subsection{Velocidad de Cierre}

La velocidad de cierre es la componente de la velocidad relativa en la dirección de la línea de visión.

\begin{definition}[Velocidad de Cierre]
    \begin{equation}
        V_c(t) = -\frac{dR}{dt} = -\frac{d}{dt}\|\mathbf{r}_{rel}\|
        \label{eq:closing_speed_def}
    \end{equation}
\end{definition}

\begin{proposition}[Expresión Vectorial de la Velocidad de Cierre]
    La velocidad de cierre puede expresarse como:
    \begin{equation}
        V_c = -\frac{\mathbf{r}_{rel} \cdot \mathbf{v}_{rel}}{R} = -\hat{\mathbf{r}}_{LOS} \cdot \mathbf{v}_{rel}
        \label{eq:closing_speed}
    \end{equation}
\end{proposition}

\begin{proof}
    Derivando la ecuación de la distancia:
    \begin{align}
        R &= \sqrt{\mathbf{r}_{rel} \cdot \mathbf{r}_{rel}} \\
        \frac{dR}{dt} &= \frac{d}{dt}\sqrt{\mathbf{r}_{rel} \cdot \mathbf{r}_{rel}} \\
        &= \frac{1}{2\sqrt{\mathbf{r}_{rel} \cdot \mathbf{r}_{rel}}} \cdot 2(\mathbf{r}_{rel} \cdot \dot{\mathbf{r}}_{rel}) \\
        &= \frac{\mathbf{r}_{rel} \cdot \mathbf{v}_{rel}}{R}
    \end{align}
    Por tanto:
    \begin{equation}
        V_c = -\frac{dR}{dt} = -\frac{\mathbf{r}_{rel} \cdot \mathbf{v}_{rel}}{R}
    \end{equation}
\end{proof}

\subsection{Velocidad Angular de la Línea de Visión}

\begin{definition}[Vector de Velocidad Angular LOS]
    La velocidad angular de rotación de la línea de visión está dada por:
    \begin{equation}
        \boldsymbol{\Omega}(t) = \frac{\mathbf{r}_{rel} \times \mathbf{v}_{rel}}{R^2}
        \label{eq:omega_los}
    \end{equation}
\end{definition}

Esta cantidad es fundamental para la navegación proporcional. El vector $\boldsymbol{\Omega}$ es perpendicular al plano formado por $\mathbf{r}_{rel}$ y $\mathbf{v}_{rel}$, y su magnitud representa la tasa de rotación angular de la LOS.

\begin{remark}
    Si $\boldsymbol{\Omega} = \mathbf{0}$, entonces $\mathbf{r}_{rel} \parallel \mathbf{v}_{rel}$, lo que implica que el misil está en un curso de colisión directo (collision course).
\end{remark}

%==============================================================================
\section{Dinámica del Objetivo}
%==============================================================================

\subsection{Modelo de Aceleración}

La aeronave objetivo se modela con una respuesta dinámica de primer orden hacia una velocidad deseada:

\begin{equation}
    \mathbf{a}_T(t) = \frac{\mathbf{v}_{deseado}(t) - \mathbf{v}_T(t)}{\tau_{resp}}
    \label{eq:target_accel}
\end{equation}

donde:
\begin{itemize}
    \item $\tau_{resp}$ es la constante de tiempo de respuesta aerodinámica (típicamente 3-5 segundos)
    \item $\mathbf{v}_{deseado}$ es la velocidad objetivo generada por la maniobra evasiva
\end{itemize}

\subsection{Generación de Maniobras Evasivas}

\subsubsection{Maniobra Espiral (Spiral)}

La maniobra espiral combina un viraje circular horizontal con ascenso constante:

\begin{align}
    \theta(t) &= \omega_{turn} \cdot t \label{eq:spiral_angle} \\
    \mathbf{d}_{spiral}(t) &= \begin{bmatrix}
        \cos(\theta(t)) \\
        \sin(\theta(t)) \\
        k_{climb}
    \end{bmatrix} \label{eq:spiral_direction} \\
    \mathbf{v}_{deseado}(t) &= V_{base} \cdot \frac{\mathbf{d}_{spiral}}{\|\mathbf{d}_{spiral}\|} \label{eq:spiral_velocity}
\end{align}

donde:
\begin{itemize}
    \item $\omega_{turn}$ es la tasa de giro (rad/s)
    \item $k_{climb}$ es la razón de ascenso normalizada
    \item $V_{base}$ es la rapidez base del objetivo
\end{itemize}

\subsubsection{Maniobra Sinusoidal (Weave)}

Oscilaciones laterales periódicas en el plano horizontal:

\begin{equation}
    \mathbf{d}_{weave}(t) = \begin{bmatrix}
        1 \\
        A \sin(\omega_{weave} t) \\
        B \sin(\frac{\omega_{weave}}{2} t)
    \end{bmatrix}
    \label{eq:weave_direction}
\end{equation}

donde $A$ y $B$ son amplitudes de oscilación lateral y vertical.

\subsubsection{Maniobra Jinking (Estocástica)}

Cambios aleatorios de dirección para romper el tracking predictivo:

\begin{equation}
    \mathbf{d}_{jink}(t) = \begin{bmatrix}
        1 \\
        A \cdot \xi_1(t) \\
        B \cdot \xi_2(t)
    \end{bmatrix}
    \label{eq:jink_direction}
\end{equation}

donde $\xi_1(t), \xi_2(t)$ son procesos estocásticos (e.g., ruido gaussiano o procesos de Ornstein-Uhlenbeck).

%==============================================================================
\section{Navegación Proporcional: Derivación Rigurosa y Análisis Físico}
%==============================================================================

La navegación proporcional (PN) es la ley de guiado más utilizada en sistemas de misiles modernos debido a su robustez, simplicidad y eficiencia energética. Su fundamento matemático se basa en el principio de que la intercepción se logra cuando la línea de visión (LOS) entre el misil y el objetivo no rota, es decir, cuando se establece un \textit{curso de colisión}.

\subsection{Principio Fundamental y Motivación Física}

\begin{theorem}[Condición Necesaria de Colisión]
Si dos objetos se mueven en el espacio y colisionan en un punto $\mathbf{P}_c$ en el tiempo $t_c$, entonces la dirección de la línea que los une permanece constante durante la aproximación, es decir:
\begin{equation}
    \frac{d}{dt}\left(\frac{\mathbf{r}_{rel}(t)}{\|\mathbf{r}_{rel}(t)\|}\right) = \mathbf{0}, \quad \forall t \in [t_0, t_c]
\end{equation}
\end{theorem}

\begin{proof}
Si ambos objetos colisionan en $\mathbf{P}_c$, entonces:
\begin{align}
    \mathbf{r}_T(t_c) &= \mathbf{P}_c \\
    \mathbf{r}_M(t_c) &= \mathbf{P}_c
\end{align}
Por tanto, $\mathbf{r}_{rel}(t_c) = \mathbf{0}$.

Para que esto ocurra, los vectores de posición deben satisfacer que existe un escalar $\lambda(t)$ tal que:
\begin{equation}
    \mathbf{r}_T(t) - \mathbf{r}_M(t) = \lambda(t)(\mathbf{r}_T(t_c) - \mathbf{r}_M(t_c))
\end{equation}
con $\lambda(t_c) = 0$. Esto implica que el vector unitario $\hat{\mathbf{r}}_{LOS} = \mathbf{r}_{rel}/\|\mathbf{r}_{rel}\|$ es constante.
\end{proof}

\subsection{Formulación Matemática Completa}

\begin{theorem}[Ley de Navegación Proporcional]
La aceleración de comando del misil es proporcional a la velocidad angular de la línea de visión y perpendicular a dicha línea:
\begin{equation}
    \mathbf{a}_{PN} = N \cdot V_c \cdot (\boldsymbol{\Omega} \times \hat{\mathbf{r}}_{LOS})
    \label{eq:pn_law_main}
\end{equation}
donde:
\begin{itemize}
    \item $N$ es la constante de navegación (ganancia), típicamente $3 \leq N \leq 5$
    \item $V_c$ es la velocidad de cierre
    \item $\boldsymbol{\Omega}$ es la velocidad angular de la LOS
    \item $\hat{\mathbf{r}}_{LOS}$ es el vector unitario de la línea de visión
\end{itemize}
\end{theorem}

\subsection{Derivación Detallada a partir de Primeros Principios}

\subsubsection{Paso 1: Cálculo de la Velocidad Angular de LOS}

Partimos de las definiciones establecidas:
\begin{align}
    \mathbf{r}_{rel} &= \mathbf{r}_T - \mathbf{r}_M \in \mathbb{R}^3 \\
    \mathbf{v}_{rel} &= \mathbf{v}_T - \mathbf{v}_M \in \mathbb{R}^3 \\
    R &= \|\mathbf{r}_{rel}\| = \sqrt{\mathbf{r}_{rel} \cdot \mathbf{r}_{rel}}
\end{align}

El vector unitario de LOS es:
\begin{equation}
    \hat{\mathbf{r}}_{LOS}(t) = \frac{\mathbf{r}_{rel}(t)}{R(t)}
\end{equation}

Su derivada temporal proporciona la velocidad angular:
\begin{align}
    \frac{d\hat{\mathbf{r}}_{LOS}}{dt} &= \frac{d}{dt}\left(\frac{\mathbf{r}_{rel}}{R}\right) \\
    &= \frac{1}{R}\frac{d\mathbf{r}_{rel}}{dt} - \mathbf{r}_{rel}\frac{1}{R^2}\frac{dR}{dt} \\
    &= \frac{\mathbf{v}_{rel}}{R} - \frac{\mathbf{r}_{rel}}{R^2}\frac{dR}{dt}
\end{align}

Sabemos que:
\begin{equation}
    \frac{dR}{dt} = \frac{\mathbf{r}_{rel} \cdot \mathbf{v}_{rel}}{R}
\end{equation}

Sustituyendo:
\begin{align}
    \frac{d\hat{\mathbf{r}}_{LOS}}{dt} &= \frac{\mathbf{v}_{rel}}{R} - \frac{\mathbf{r}_{rel}}{R^3}(\mathbf{r}_{rel} \cdot \mathbf{v}_{rel}) \\
    &= \frac{1}{R}\left[\mathbf{v}_{rel} - \frac{\mathbf{r}_{rel}(\mathbf{r}_{rel} \cdot \mathbf{v}_{rel})}{R^2}\right] \\
    &= \frac{1}{R}\left[\mathbf{v}_{rel} - \hat{\mathbf{r}}_{LOS}(\hat{\mathbf{r}}_{LOS} \cdot \mathbf{v}_{rel})\right] \\
    &= \frac{1}{R}\mathbf{v}_{rel}^{\perp}
\end{align}

donde $\mathbf{v}_{rel}^{\perp}$ es la componente de $\mathbf{v}_{rel}$ perpendicular a $\mathbf{r}_{rel}$.

\subsubsection{Paso 2: Relación con el Producto Vectorial}

Consideremos el producto vectorial:
\begin{equation}
    \mathbf{r}_{rel} \times \mathbf{v}_{rel}
\end{equation}

Este vector es perpendicular al plano formado por $\mathbf{r}_{rel}$ y $\mathbf{v}_{rel}$, y su magnitud es:
\begin{equation}
    \|\mathbf{r}_{rel} \times \mathbf{v}_{rel}\| = R \cdot \|\mathbf{v}_{rel}\| \sin(\alpha)
\end{equation}
donde $\alpha$ es el ángulo entre los dos vectores.

Definimos la velocidad angular como:
\begin{equation}
    \boldsymbol{\Omega} = \frac{\mathbf{r}_{rel} \times \mathbf{v}_{rel}}{R^2}
    \label{eq:omega_def_detailed}
\end{equation}

Este vector tiene las siguientes propiedades físicas:
\begin{enumerate}
    \item $\boldsymbol{\Omega} \perp \mathbf{r}_{rel}$ (perpendicular a la LOS)
    \item $\boldsymbol{\Omega} \perp \mathbf{v}_{rel}$ (perpendicular a la velocidad relativa)
    \item $\|\boldsymbol{\Omega}\| = \frac{\|\mathbf{v}_{rel}\|\sin(\alpha)}{R}$ (tasa de rotación angular)
\end{enumerate}

\subsubsection{Paso 3: Expansión del Término de Aceleración PN}

Ahora calculamos el producto vectorial triple:
\begin{equation}
    \boldsymbol{\Omega} \times \hat{\mathbf{r}}_{LOS} = \frac{\mathbf{r}_{rel} \times \mathbf{v}_{rel}}{R^2} \times \frac{\mathbf{r}_{rel}}{R}
\end{equation}

Utilizando la identidad del producto vectorial triple:
\begin{equation}
    \mathbf{a} \times (\mathbf{b} \times \mathbf{c}) = \mathbf{b}(\mathbf{a} \cdot \mathbf{c}) - \mathbf{c}(\mathbf{a} \cdot \mathbf{b})
\end{equation}

Tenemos:
\begin{align}
    (\mathbf{r}_{rel} \times \mathbf{v}_{rel}) \times \mathbf{r}_{rel} &= \mathbf{r}_{rel}\underbrace{((\mathbf{r}_{rel} \times \mathbf{v}_{rel}) \cdot \mathbf{r}_{rel})}_{=0} - \mathbf{r}_{rel}((\mathbf{r}_{rel} \times \mathbf{v}_{rel}) \cdot \mathbf{r}_{rel}) \\
    & \quad - \mathbf{r}_{rel}\times\mathbf{v}_{rel} \cdot \mathbf{r}_{rel}\\
    &= -\mathbf{r}_{rel}(\mathbf{r}_{rel} \cdot \mathbf{v}_{rel}) + \mathbf{v}_{rel}(\mathbf{r}_{rel} \cdot \mathbf{r}_{rel}) \\
    &= R^2\mathbf{v}_{rel} - \mathbf{r}_{rel}(\mathbf{r}_{rel} \cdot \mathbf{v}_{rel})
\end{align}

Dividiendo por $R^3$:
\begin{align}
    \boldsymbol{\Omega} \times \hat{\mathbf{r}}_{LOS} &= \frac{1}{R}\left[\mathbf{v}_{rel} - \frac{\mathbf{r}_{rel}(\mathbf{r}_{rel} \cdot \mathbf{v}_{rel})}{R^2}\right] \\
    &= \frac{1}{R}\mathbf{v}_{rel}^{\perp}
    \label{eq:pn_perpendicular_detailed}
\end{align}

donde $\mathbf{v}_{rel}^{\perp}$ es la componente de $\mathbf{v}_{rel}$ perpendicular a $\mathbf{r}_{rel}$.

\subsection{Interpretación Geométrica}

La ley PN genera una aceleración que:
\begin{enumerate}
    \item Es perpendicular a la línea de visión
    \item Tiende a anular la rotación angular de la LOS ($\boldsymbol{\Omega} \to \mathbf{0}$)
    \item Conduce al misil hacia un curso de colisión
\end{enumerate}

\begin{proposition}[Condición de Colisión]
    Si $\boldsymbol{\Omega}(t) = \mathbf{0}$ para todo $t \in [t_0, t_f]$, entonces el misil está en curso de colisión directo.
\end{proposition}

\subsection{Expansión Explícita de la Ley PN}

Sustituyendo la definición de $\boldsymbol{\Omega}$:

\begin{equation}
    \mathbf{a}_{PN} = N \cdot V_c \cdot \left[\frac{\mathbf{r}_{rel} \times \mathbf{v}_{rel}}{R^2} \times \frac{\mathbf{r}_{rel}}{R}\right]
    \label{eq:pn_explicit}
\end{equation}

En términos de las variables de estado:
\begin{align}
    \mathbf{r}_{rel} &= \mathbf{r}_T - \mathbf{r}_M \\
    \mathbf{v}_{rel} &= \mathbf{v}_T - \mathbf{v}_M \\
    R &= \|\mathbf{r}_T - \mathbf{r}_M\| \\
    V_c &= -\frac{(\mathbf{r}_T - \mathbf{r}_M) \cdot (\mathbf{v}_T - \mathbf{v}_M)}{R}
\end{align}

%==============================================================================
\section{Guiado Predictivo (Lead Guidance)}
%==============================================================================

El guiado predictivo complementa la PN al estimar el punto futuro de intercepción.

\subsection{Tiempo de Vuelo Estimado}

\begin{definition}[Tiempo hasta el Impacto]
    El tiempo estimado hasta la intercepción se calcula como:
    \begin{equation}
        t_{go}(t) = \frac{R(t)}{V_{rel}(t)}
        \label{eq:time_to_go}
    \end{equation}
    donde:
    \begin{equation}
        V_{rel} = \|\mathbf{v}_M\| + \|\mathbf{v}_T\| \quad \text{(aproximación conservadora)}
    \end{equation}
\end{definition}

\subsection{Punto de Impacto Predicho}

Asumiendo movimiento rectilíneo uniforme del objetivo en el intervalo $[t, t + t_{go}]$:

\begin{equation}
    \mathbf{P}_{impacto}(t) = \mathbf{r}_T(t) + \mathbf{v}_T(t) \cdot t_{go}(t)
    \label{eq:impact_point}
\end{equation}

\subsection{Comando de Aceleración Lead}

El vector de error de lead es:
\begin{equation}
    \mathbf{e}_{lead}(t) = \mathbf{P}_{impacto}(t) - \mathbf{r}_M(t)
    \label{eq:lead_error}
\end{equation}

La aceleración de comando se genera mediante un controlador proporcional:
\begin{equation}
    \mathbf{a}_{lead} = K_{lead} \cdot \frac{\mathbf{e}_{lead}}{\|\mathbf{e}_{lead}\|} \cdot \left(\|\mathbf{v}_{deseado}\| - \|\mathbf{v}_M\|\right)
    \label{eq:lead_accel}
\end{equation}

donde $K_{lead}$ es la ganancia del controlador de lead.

%==============================================================================
\section{Control Híbrido y Fusión de Estrategias}
%==============================================================================

El sistema implementa un controlador híbrido que combina múltiples estrategias según la fase de vuelo.

\subsection{Función de Ponderación Temporal}

\begin{definition}[Peso de Navegación Proporcional]
    \begin{equation}
        w_{PN}(t, R) = \begin{cases}
            w_{bias} & \text{si } t \leq t_{warmup} \\
            w_{bias} + (1 - w_{bias})\frac{t - t_{warmup}}{t_{blend}} & \text{si } t_{warmup} < t \leq t_{warmup} + t_{blend} \\
            1 & \text{si } t > t_{warmup} + t_{blend} \text{ o } R < R_{terminal}
        \end{cases}
        \label{eq:pn_weight}
    \end{equation}
\end{definition}

\subsection{Aceleración Total del Misil}

La aceleración total es una combinación convexa de las estrategias:

\begin{equation}
    \mathbf{a}_M(t) = w_{PN}(t, R) \cdot \mathbf{a}_{guided} + (1 - w_{PN}(t, R)) \cdot \mathbf{a}_{maneuver}
    \label{eq:total_accel}
\end{equation}

donde:
\begin{align}
    \mathbf{a}_{guided} &= \mathbf{a}_{PN} + \mathbf{a}_{speed} + \mathbf{a}_{lead} \\
    \mathbf{a}_{speed} &= K_{speed}(V_{cruise} - \|\mathbf{v}_M\|)\hat{\mathbf{v}}_M \quad \text{(control de velocidad)} \\
    \mathbf{a}_{maneuver} &= \mathbf{a}_{loft} + \mathbf{a}_{weave} \quad \text{(maniobras pre-programadas)}
\end{align}

\subsection{Maniobras de Boost}

Durante la fase inicial ($t < t_{boost}$), el misil realiza una maniobra de "loft" (ganancia de altura):

\begin{align}
    \theta_{loft} &= \frac{\pi}{180} \cdot \theta_{degrees} \\
    \hat{\mathbf{u}}_{loft} &= \cos(\theta_{loft})\hat{\mathbf{v}}_M + \sin(\theta_{loft})\hat{\mathbf{z}} \\
    \mathbf{a}_{loft} &= a_{boost} \cdot \hat{\mathbf{u}}_{loft}
    \label{eq:loft_maneuver}
\end{align}

\subsection{Limitación de Aceleración}

Para respetar las restricciones físicas del misil:

\begin{equation}
    \mathbf{a}_M^{final} = \begin{cases}
        \mathbf{a}_M & \text{si } \|\mathbf{a}_M\| \leq a_{max} \\
        a_{max} \cdot \frac{\mathbf{a}_M}{\|\mathbf{a}_M\|} & \text{si } \|\mathbf{a}_M\| > a_{max}
    \end{cases}
    \label{eq:accel_clamp}
\end{equation}

%==============================================================================
\section{Integración Numérica: Método Runge-Kutta de Cuarto Orden}
%==============================================================================

La solución del sistema \eqref{eq:full_system} requiere métodos numéricos de alta precisión.

\subsection{Formulación del Problema de Valor Inicial}

Dado el sistema de EDOs:
\begin{equation}
    \frac{d\mathbf{S}}{dt} = \mathbf{F}(t, \mathbf{S}), \quad \mathbf{S}(t_0) = \mathbf{S}_0
    \label{eq:ivp}
\end{equation}

Buscamos aproximar la solución $\mathbf{S}(t)$ en puntos discretos $t_n = t_0 + n \cdot h$, donde $h$ es el paso de tiempo.

\subsection{Método de Euler (Referencia)}

El método de Euler de primer orden es:
\begin{equation}
    \mathbf{S}_{n+1} = \mathbf{S}_n + h \cdot \mathbf{F}(t_n, \mathbf{S}_n)
    \label{eq:euler}
\end{equation}

con error de truncamiento local $\mathcal{O}(h^2)$ y error global $\mathcal{O}(h)$.

\subsection{Método Runge-Kutta de Orden 4 (RK4)}

\begin{theorem}[Algoritmo RK4]
    El método RK4 aproxima la solución mediante la combinación ponderada de cuatro pendientes:
    \begin{align}
        \mathbf{k}_1 &= \mathbf{F}(t_n, \mathbf{S}_n) \label{eq:rk4_k1} \\
        \mathbf{k}_2 &= \mathbf{F}\left(t_n + \frac{h}{2}, \mathbf{S}_n + \frac{h}{2}\mathbf{k}_1\right) \label{eq:rk4_k2} \\
        \mathbf{k}_3 &= \mathbf{F}\left(t_n + \frac{h}{2}, \mathbf{S}_n + \frac{h}{2}\mathbf{k}_2\right) \label{eq:rk4_k3} \\
        \mathbf{k}_4 &= \mathbf{F}(t_n + h, \mathbf{S}_n + h\mathbf{k}_3) \label{eq:rk4_k4}
    \end{align}
    
    El estado en el siguiente paso es:
    \begin{equation}
        \mathbf{S}_{n+1} = \mathbf{S}_n + \frac{h}{6}\left(\mathbf{k}_1 + 2\mathbf{k}_2 + 2\mathbf{k}_3 + \mathbf{k}_4\right)
        \label{eq:rk4_step}
    \end{equation}
\end{theorem}

\subsection{Interpretación Geométrica}

El método RK4 evalúa la pendiente (derivada) en cuatro puntos:
\begin{enumerate}
    \item $\mathbf{k}_1$: pendiente al inicio del intervalo
    \item $\mathbf{k}_2$: pendiente en el punto medio, usando $\mathbf{k}_1$ para llegar allí
    \item $\mathbf{k}_3$: pendiente en el punto medio, usando $\mathbf{k}_2$ (corrección mejorada)
    \item $\mathbf{k}_4$: pendiente al final del intervalo, usando $\mathbf{k}_3$
\end{enumerate}

La combinación $\frac{1}{6}(1, 2, 2, 1)$ es una cuadratura de Simpson que minimiza el error de truncamiento.

\subsection{Análisis de Error}

\begin{theorem}[Error de Truncamiento Local]
    El error de truncamiento local del método RK4 es:
    \begin{equation}
        \tau_{n+1} = \mathbf{S}(t_{n+1}) - \mathbf{S}_{n+1} = \mathcal{O}(h^5)
        \label{eq:rk4_local_error}
    \end{equation}
\end{theorem}

\begin{corollary}[Error Global]
    Para un intervalo de integración finito $[t_0, T]$, el error global es:
    \begin{equation}
        \|\mathbf{S}(T) - \mathbf{S}_N\| = \mathcal{O}(h^4)
        \label{eq:rk4_global_error}
    \end{equation}
\end{corollary}

\subsection{Comparación Cuantitativa: Euler vs RK4}

Para un paso de tiempo $h = 0.05$ segundos:

\begin{center}
\begin{tabular}{|l|c|c|}
\hline
\textbf{Métrica} & \textbf{Euler} & \textbf{RK4} \\
\hline
Error local & $\mathcal{O}(h^2) \approx 2.5 \times 10^{-3}$ & $\mathcal{O}(h^5) \approx 3 \times 10^{-7}$ \\
Error global & $\mathcal{O}(h) \approx 5 \times 10^{-2}$ & $\mathcal{O}(h^4) \approx 6 \times 10^{-6}$ \\
Estabilidad numérica & Inestable para $h > h_{crit}$ & Estable para $h \ll 1$ \\
Conservación de energía & Deriva sistemática & Conservación $< 10^{-6}$ \\
\hline
\end{tabular}
\end{center}

\subsection{Implementación Computacional}

El algoritmo RK4 aplicado a nuestro sistema es:

\begin{align}
    \mathbf{k}_1 &= \begin{bmatrix} \mathbf{v}_T^{(n)} \\ \mathbf{a}_T(t_n, \mathbf{r}_T^{(n)}, \mathbf{v}_T^{(n)}) \\ \mathbf{v}_M^{(n)} \\ \mathbf{a}_M(t_n, \mathbf{S}^{(n)}) \end{bmatrix} \\
    \mathbf{k}_2 &= \mathbf{F}\left(t_n + \frac{h}{2}, \mathbf{S}_n + \frac{h}{2}\mathbf{k}_1\right) \\
    \mathbf{k}_3 &= \mathbf{F}\left(t_n + \frac{h}{2}, \mathbf{S}_n + \frac{h}{2}\mathbf{k}_2\right) \\
    \mathbf{k}_4 &= \mathbf{F}(t_n + h, \mathbf{S}_n + h\mathbf{k}_3)
\end{align}

donde cada evaluación de $\mathbf{F}$ requiere:
\begin{itemize}
    \item Desempaquetar el vector de estado
    \item Calcular la cinemática relativa ($\mathbf{r}_{rel}$, $\mathbf{v}_{rel}$, $R$, $V_c$, $\boldsymbol{\Omega}$)
    \item Evaluar la ley de guiado PN
    \item Calcular la predicción lead
    \item Aplicar fusión de controladores
    \item Re-empaquetar las derivadas
\end{itemize}

\subsection{Implementación en Python}

La implementación computacional traduce las ecuaciones matemáticas a código ejecutable. A continuación se muestra cómo se implementan los componentes clave del sistema.

\subsubsection{Desempaquetado del Vector de Estado}

El vector de estado $\mathbf{S} \in \mathbb{R}^{12}$ se desempaqueta en sus componentes:

\begin{lstlisting}[language=Python, caption={Desempaquetado del vector de estado}, label={lst:unpack}]
def unpack_state(state: np.ndarray) -> tuple:
    """Split the 12-element state vector into component vectors."""
    aircraft_pos = state[0:3]  # r_T
    aircraft_vel = state[3:6]  # v_T
    missile_pos = state[6:9]   # r_M
    missile_vel = state[9:12]  # v_M
    return aircraft_pos, aircraft_vel, missile_pos, missile_vel
\end{lstlisting}

Esto corresponde directamente a la descomposición:
\begin{equation}
    \mathbf{S} = [\mathbf{r}_T^T, \mathbf{v}_T^T, \mathbf{r}_M^T, \mathbf{v}_M^T]^T
\end{equation}

\subsubsection{Cálculo de la Velocidad de Cierre}

La velocidad de cierre definida en la ecuación \eqref{eq:closing_speed} se implementa como:

\begin{lstlisting}[language=Python, caption={Implementación de velocidad de cierre}]
los = target_pos - missile_pos  # r_rel
distance = norm(los)             # R = ||r_rel||
los_unit = los / distance        # r_LOS_hat

relative_velocity = target_vel - missile_vel  # v_rel
closing_speed = -np.dot(relative_velocity, los_unit)  # V_c
\end{lstlisting}

que implementa:
\begin{equation}
    V_c = -\frac{\mathbf{r}_{rel} \cdot \mathbf{v}_{rel}}{R} = -\hat{\mathbf{r}}_{LOS} \cdot \mathbf{v}_{rel}
\end{equation}

\subsubsection{Velocidad Angular de la Línea de Visión}

El cálculo de $\boldsymbol{\Omega}$ según \eqref{eq:omega_def_detailed} se traduce a:

\begin{lstlisting}[language=Python, caption={Cálculo de velocidad angular LOS}]
omega = np.cross(los, relative_velocity) / max(distance ** 2, EPS)
\end{lstlisting}

Implementando:
\begin{equation}
    \boldsymbol{\Omega} = \frac{\mathbf{r}_{rel} \times \mathbf{v}_{rel}}{R^2}
\end{equation}

Nota: el término \texttt{max(distance ** 2, EPS)} previene división por cero, añadiendo un epsilon de seguridad $\epsilon = 10^{-9}$.

\subsubsection{Comando de Aceleración PN}

La ley completa de navegación proporcional \eqref{eq:pn_law_main} se implementa como:

\begin{lstlisting}[language=Python, caption={Implementación de la ley PN}, label={lst:pn_implementation}]
def guidance_acceleration(self, missile_pos, missile_vel, 
                         target_pos, target_vel):
    # Geometria relativa
    los = target_pos - missile_pos
    distance = norm(los)
    los_unit = los / distance
    
    # Cinematica relativa  
    relative_velocity = target_vel - missile_vel
    closing_speed = -np.dot(relative_velocity, los_unit)
    effective_closing = max(closing_speed, self.min_closing_speed)
    
    # Velocidad angular de LOS
    omega = np.cross(los, relative_velocity) / max(distance ** 2, EPS)
    
    # Comando PN
    pn_cmd = self.nav_gain * effective_closing * np.cross(omega, los_unit)
    
    return pn_cmd
\end{lstlisting}

Este código implementa exactamente:
\begin{equation}
    \mathbf{a}_{PN} = N \cdot V_c \cdot (\boldsymbol{\Omega} \times \hat{\mathbf{r}}_{LOS})
\end{equation}

\subsubsection{Integración RK4}

El método Runge-Kutta de cuarto orden descrito en \eqref{eq:rk4_step} se implementa como:

\begin{lstlisting}[language=Python, caption={Implementación del método RK4}]
def rk4_step(func, t, state, dt):
    """Single step of 4th-order Runge-Kutta integration."""
    k1 = func(t, state)
    k2 = func(t + 0.5 * dt, state + 0.5 * dt * k1)
    k3 = func(t + 0.5 * dt, state + 0.5 * dt * k2)
    k4 = func(t + dt, state + dt * k3)
    return state + (dt / 6.0) * (k1 + 2 * k2 + 2 * k3 + k4)
\end{lstlisting}

que corresponde exactamente a:
\begin{align}
    \mathbf{k}_1 &= \mathbf{F}(t_n, \mathbf{S}_n) \\
    \mathbf{k}_2 &= \mathbf{F}(t_n + \frac{h}{2}, \mathbf{S}_n + \frac{h}{2}\mathbf{k}_1) \\
    \mathbf{k}_3 &= \mathbf{F}(t_n + \frac{h}{2}, \mathbf{S}_n + \frac{h}{2}\mathbf{k}_2) \\
    \mathbf{k}_4 &= \mathbf{F}(t_n + h, \mathbf{S}_n + h\mathbf{k}_3) \\
    \mathbf{S}_{n+1} &= \mathbf{S}_n + \frac{h}{6}(\mathbf{k}_1 + 2\mathbf{k}_2 + 2\mathbf{k}_3 + \mathbf{k}_4)
\end{align}

\subsubsection{Función de Derivadas del Sistema}

La función que calcula $\mathbf{F}(t, \mathbf{S})$ para el sistema completo:

\begin{lstlisting}[language=Python, caption={Función de derivadas del sistema}]
def derivatives(time: float, state_vec: np.ndarray) -> np.ndarray:
    # Desempaquetar estado
    a_pos, a_vel, m_pos, m_vel = unpack_state(state_vec)
    
    # Calcular aceleraciones
    a_acc = aircraft.acceleration(time, a_pos, a_vel)
    m_acc = missile.guidance_acceleration(time, m_pos, m_vel, 
                                         a_pos, a_vel)
    
    # Empaquetar derivadas: d/dt[r_T, v_T, r_M, v_M] = [v_T, a_T, v_M, a_M]
    return pack_state(a_vel, a_acc, m_vel, m_acc)
\end{lstlisting}

Implementa:
\begin{equation}
    \frac{d\mathbf{S}}{dt} = \mathbf{F}(t, \mathbf{S}) = [\mathbf{v}_T, \mathbf{a}_T, \mathbf{v}_M, \mathbf{a}_M]^T
\end{equation}

\subsubsection{Bucle Principal de Simulación}

El bucle de integración completo que resuelve el sistema de EDOs:

\begin{lstlisting}[language=Python, caption={Bucle principal de simulación}]
def simulate(aircraft, missile, initial_state, dt, duration, 
            intercept_tolerance, method="rk4"):
    stepper = rk4_step if method == "rk4" else euler_step
    
    current_time = 0.0
    current_state = initial_state.copy()
    intercept_time = None
    
    while True:
        # Calcular distancia actual
        a_pos, _, m_pos, _ = unpack_state(current_state)
        distance = relative_distance(a_pos, m_pos)
        
        # Verificar condicion de interception
        if distance <= intercept_tolerance:
            intercept_time = current_time
            break
        if current_time >= duration:
            break
        
        # Integrar un paso
        next_state = stepper(derivatives, current_time, 
                            current_state, dt)
        
        # Avanzar tiempo
        current_time = min(current_time + dt, duration)
        current_state = next_state
    
    return intercept_time, current_state
\end{lstlisting}

Este código implementa el problema de valor inicial completo, resolviendo iterativamente el sistema desde $t_0$ hasta que $R(t) \leq R_{tol}$ o $t \geq T_{max}$.

%==============================================================================
\section{Análisis de Estabilidad del Sistema}
%==============================================================================

\subsection{Linealización del Sistema}

Para analizar la estabilidad local, consideramos pequeñas perturbaciones alrededor de una trayectoria nominal.

Sea $\mathbf{S}(t) = \mathbf{S}_{nom}(t) + \delta\mathbf{S}(t)$, donde $\|\delta\mathbf{S}\| \ll 1$.

Expandiendo en serie de Taylor alrededor de la trayectoria nominal:
\begin{equation}
    \frac{d\delta\mathbf{S}}{dt} = \mathbf{J}(t) \cdot \delta\mathbf{S} + \mathcal{O}(\|\delta\mathbf{S}\|^2)
\end{equation}

donde $\mathbf{J}$ es la matriz Jacobiana definida en \eqref{eq:jacobian_structure}.

\subsubsection{Sistema Lineal Variante en el Tiempo}

El sistema linealizado es un sistema LTV (Linear Time-Varying):
\begin{equation}
    \frac{d\delta\mathbf{S}}{dt} = \mathbf{J}(t) \delta\mathbf{S}
    \label{eq:linearized_system}
\end{equation}

La estabilidad de este sistema se determina mediante el análisis de la matriz de transición de estado $\boldsymbol{\Phi}(t, t_0)$, que satisface:
\begin{align}
    \frac{d\boldsymbol{\Phi}}{dt} &= \mathbf{J}(t)\boldsymbol{\Phi}(t, t_0) \\
    \boldsymbol{\Phi}(t_0, t_0) &= \mathbf{I}_{12}
\end{align}

\subsection{Condiciones de Estabilidad}

\begin{theorem}[Criterio de Estabilidad de Lyapunov para Sistemas LTV]
El sistema linealizado \eqref{eq:linearized_system} es asintóticamente estable si y solo si:
\begin{equation}
    \lim_{t \to \infty} \|\boldsymbol{\Phi}(t, t_0)\| = 0
\end{equation}
para todo $t_0 \geq 0$.
\end{theorem}

Para sistemas autónomos ($\mathbf{J}$ constante), esto se reduce a:
\begin{equation}
    \text{Re}(\lambda_i(\mathbf{J})) < 0, \quad \forall i \in \{1, \ldots, 12\}
\end{equation}

\subsection{Análisis de Lyapunov para el Sistema No Lineal}

Para el sistema completo no lineal, utilizamos el método directo de Lyapunov.

\subsubsection{Función Candidata de Lyapunov}

Definimos la función candidata de Lyapunov como la energía posicional relativa:
\begin{equation}
    V(\mathbf{S}) = \frac{1}{2}R^2 = \frac{1}{2}\|\mathbf{r}_T - \mathbf{r}_M\|^2
    \label{eq:lyapunov_function}
\end{equation}

Esta función es claramente:
\begin{enumerate}
    \item Definida positiva: $V(\mathbf{S}) > 0$ para $\mathbf{r}_T \neq \mathbf{r}_M$
    \item Radialmente no acotada: $V(\mathbf{S}) \to \infty$ cuando $R \to \infty$
    \item Diferenciable en todo el dominio excepto en $\mathbf{r}_T = \mathbf{r}_M$
\end{enumerate}

\subsubsection{Derivada Temporal de la Función de Lyapunov}

Calculamos $\dot{V}$ a lo largo de las trayectorias del sistema:
\begin{align}
    \dot{V} &= \frac{dV}{dt} = \frac{d}{dt}\left(\frac{1}{2}R^2\right) \\
    &= R\frac{dR}{dt} \\
    &= R \cdot \frac{\mathbf{r}_{rel} \cdot \mathbf{v}_{rel}}{R} \\
    &= \mathbf{r}_{rel} \cdot \mathbf{v}_{rel} \\
    &= (\mathbf{r}_T - \mathbf{r}_M) \cdot (\mathbf{v}_T - \mathbf{v}_M)
    \label{eq:lyapunov_derivative}
\end{align}

En términos de la velocidad de cierre:
\begin{equation}
    \dot{V} = -R \cdot V_c
\end{equation}

\begin{theorem}[Condición de Captura Garantizada]
Si la velocidad de cierre satisface:
\begin{equation}
    V_c(t) \geq V_{c,min} > 0, \quad \forall t \in [t_0, \infty)
\end{equation}
entonces la intercepción está garantizada en tiempo finito.
\end{theorem}

\begin{proof}
De \eqref{eq:lyapunov_derivative}:
\begin{equation}
    \dot{V} = -R \cdot V_c \leq -R \cdot V_{c,min} < 0
\end{equation}

Esto implica que $V(t)$ es estrictamente decreciente. Integrando:
\begin{align}
    \int_{t_0}^{t_f} \dot{V}(\tau) d\tau &= V(t_f) - V(t_0) \\
    -\int_{t_0}^{t_f} R(\tau) V_c(\tau) d\tau &\leq -V_{c,min}\int_{t_0}^{t_f} R(\tau) d\tau
\end{align}

Para que $V(t_f) \to 0$ (intercepción), la integral de distancia debe ser finita, lo cual está garantizado cuando $V_c$ está acotada inferiormente por una constante positiva.
\end{proof}

\subsubsection{Condición Necesaria y Suficiente para Intercepción}

\begin{theorem}[Teorema de Capturabilidad]
Dada una ley de guiado $\mathbf{a}_M = \mathbf{g}(\mathbf{S}, t)$, la intercepción es posible si y solo si:
\begin{equation}
    \lim_{t \to t_f} \int_0^{t_f} V_c(\tau) d\tau = \infty
\end{equation}
lo cual es equivalente a:
\begin{equation}
    \lim_{t \to t_f} R(t) = 0
\end{equation}
\end{theorem}

\subsection{Región de Captura}

\subsubsection{Definición del Conjunto de Captura}

\begin{definition}[Región de Captura]
Para una ley de guiado dada y parámetros del misil fijos, la región de captura $\mathcal{C} \subset \mathbb{R}^{12}$ se define como:
\begin{equation}
    \mathcal{C} = \{\mathbf{S}_0 \in \mathbb{R}^{12} : \exists t_f < \infty \text{ tal que } R(t_f; \mathbf{S}_0) \leq R_{tol}\}
\end{equation}
\end{definition}

Esta región depende de:
\begin{itemize}
    \item La constante de navegación $N$
    \item La aceleración máxima del misil $a_{max}$
    \item La ventaja de velocidad $\Delta V = V_M - V_T$
    \item La geometría inicial del compromiso (aspect angle)
\end{itemize}

\subsubsection{Estimación de la Región de Captura}

Para navegación proporcional, existe una estimación clásica:

\begin{theorem}[Cono de Captura]
Para PN pura con $N \geq 3$, la región de captura contiene al menos el cono:
\begin{equation}
    \mathcal{C} \supseteq \left\{\mathbf{S}_0 : \frac{\|\mathbf{v}_{rel}^{\perp}\|}{V_c} < \sqrt{\frac{N-1}{N+1}}\right\}
\end{equation}
\end{theorem}

Esta condición establece un límite superior en la componente perpendicular de velocidad relativa para garantizar captura.

\subsection{Análisis de Puntos Críticos}

\subsubsection{Identificación de Puntos de Equilibrio}

Los puntos de equilibrio del sistema satisfacen $\mathbf{F}(\mathbf{S}^*) = \mathbf{0}$.

Para nuestro sistema dinámico:
\begin{align}
    \mathbf{v}_T^* &= \mathbf{0} \\
    \mathbf{v}_M^* &= \mathbf{0} \\
    \mathbf{a}_T(\mathbf{r}_T^*, \mathbf{0}) &= \mathbf{0} \\
    \mathbf{a}_M(\mathbf{r}_M^*, \mathbf{0}, \mathbf{r}_T^*, \mathbf{0}) &= \mathbf{0}
\end{align}

En el caso de PN pura:
\begin{equation}
    \mathbf{a}_{PN} = N V_c (\boldsymbol{\Omega} \times \hat{\mathbf{r}}_{LOS})
\end{equation}

Con $\mathbf{v}_T^* = \mathbf{v}_M^* = \mathbf{0}$:
\begin{align}
    \mathbf{v}_{rel}^* &= \mathbf{0} \\
    V_c^* &= 0 \\
    \boldsymbol{\Omega}^* &= \mathbf{0} \\
    \mathbf{a}_{PN}^* &= \mathbf{0}
\end{align}

\begin{proposition}[Conjunto de Equilibrios]
El conjunto de puntos de equilibrio forma una variedad 6-dimensional:
\begin{equation}
    \mathcal{E} = \{(\mathbf{r}_T, \mathbf{0}, \mathbf{r}_M, \mathbf{0}) : \mathbf{r}_T, \mathbf{r}_M \in \mathbb{R}^3\}
\end{equation}
\end{proposition}

Estos equilibrios son todos inestables (neutralmente estables sin atracción), ya que cualquier perturbación de velocidad inicia el movimiento.

\subsection{Bifurcaciones y Comportamiento Crítico}

\subsubsection{Bifurcación en Función de $N$}

La constante de navegación $N$ actúa como parámetro de bifurcación:

\begin{itemize}
    \item $N < 2$: Sistema subamortiguado, oscilaciones divergentes
    \item $N = 2$: Bifurcación crítica (valor mínimo teórico)
    \item $2 < N < 5$: Régimen óptimo, convergencia suave
    \item $N > 5$: Sobreamortiguado, demanda excesiva de aceleración
\end{itemize}

\subsubsection{Análisis de Sensibilidad}

La sensibilidad del tiempo de intercepción respecto a $N$ se puede expresar como:
\begin{equation}
    \frac{\partial t_f}{\partial N} = -\int_0^{t_f} \frac{V_c}{N} \cdot \frac{\partial R}{\partial \mathbf{a}_M} \cdot \frac{\partial \mathbf{a}_M}{\partial N} dt
\end{equation}

Esta derivada es negativa para $N \in [3, 5]$, indicando que incrementar $N$ reduce el tiempo de intercepción, pero con rendimientos marginales decrecientes.

\subsection{Teorema de Existencia y Unicidad}

\begin{theorem}[Existencia y Unicidad de Soluciones]
Dado el sistema \eqref{eq:full_system} con condiciones iniciales $\mathbf{S}(t_0) = \mathbf{S}_0 \in \mathbb{R}^{12}$ y asumiendo que las funciones de aceleración $\mathbf{a}_T$ y $\mathbf{a}_M$ son localmente Lipschitz continuas en $\mathbf{S}$, entonces:
\begin{enumerate}
    \item Existe al menos una solución $\mathbf{S}(t)$ definida en un intervalo $[t_0, t_0 + T]$ para algún $T > 0$.
    \item La solución es única.
    \item La solución depende continuamente de las condiciones iniciales.
\end{enumerate}
\end{theorem}

\begin{proof}
Aplicamos el teorema de Picard-Lindelöf. Las funciones $\mathbf{a}_T$ y $\mathbf{a}_M$ son composiciones de:
\begin{itemize}
    \item Productos escalares (Lipschitz)
    \item Productos vectoriales (Lipschitz)
    \item Normalización $\mathbf{v}/\|\mathbf{v}\|$ (Lipschitz fuera de $\mathbf{v} = \mathbf{0}$)
    \item Funciones trigonométricas (Lipschitz)
\end{itemize}

La singularidad potencial en $\mathbf{r}_{rel} = \mathbf{0}$ se evita mediante la condición de intercepción $R \leq R_{tol} > 0$, que finaliza la integración antes de alcanzar la singularidad.

La continuidad respecto a las condiciones iniciales se sigue del teorema fundamental sobre ODEs.
\end{proof}

%==============================================================================
\section{Criterios de Intercepción}
%==============================================================================

\subsection{Condición de Éxito}

La intercepción se considera exitosa si:
\begin{equation}
    \exists t_f < \infty : R(t_f) \leq R_{tol}
    \label{eq:intercept_condition}
\end{equation}

donde $R_{tol}$ es el radio de tolerancia (típicamente 50-100 metros para espoletas de proximidad).

\subsection{Miss Distance}

Si no hay intercepción directa, se define la distancia de fallo como:
\begin{equation}
    d_{miss} = \min_{t \in [0, T]} R(t)
    \label{eq:miss_distance}
\end{equation}

%==============================================================================
\section{Métricas de Desempeño}
%==============================================================================

\subsection{Tiempo de Intercepción}

\begin{equation}
    t_{intercept} = \inf\{t \geq 0 : R(t) \leq R_{tol}\}
\end{equation}

\subsection{Demanda de Aceleración}

La aceleración máxima requerida:
\begin{equation}
    a_{max} = \max_{t \in [0, t_f]} \|\mathbf{a}_M(t)\|
\end{equation}

expresada en G-forces:
\begin{equation}
    G_{max} = \frac{a_{max}}{g}, \quad g = 9.81 \, \text{m/s}^2
\end{equation}

\subsection{Eficiencia Energética}

El trabajo total realizado por el sistema de propulsión:
\begin{equation}
    W = \int_0^{t_f} \|\mathbf{a}_M(t)\| \cdot \|\mathbf{v}_M(t)\| \, dt
\end{equation}

%==============================================================================
\section{Resultados Numéricos y Desarrollo del Ejemplo}
%==============================================================================

\subsection{Parámetros de Simulación}

Antes de presentar los resultados, especificaremos con precisión todos los parámetros utilizados en la simulación numérica.

\subsubsection{Configuración del Integrador}

\begin{align}
    h &= 0.05 \text{ s} \quad \text{(paso de tiempo)} \\
    T_{max} &= 120 \text{ s} \quad \text{(duración máxima)} \\
    R_{tol} &= 50 \text{ m} \quad \text{(tolerancia de intercepción)} \\
    \text{Método} &= \text{RK4}
\end{align}

\subsubsection{Condiciones Iniciales}

\textbf{Objetivo (Aeronave):}
\begin{align}
    \mathbf{r}_T(0) &= \begin{bmatrix} 20\,000 \\ 0 \\ 7\,000 \end{bmatrix} \text{ m} \\
    \mathbf{v}_T(0) &= \begin{bmatrix} 250 \\ 0 \\ 0 \end{bmatrix} \text{ m/s} \\
    V_{base}^T &= 250 \text{ m/s} \approx 900 \text{ km/h} \\
    \tau_{resp} &= 5.0 \text{ s}
\end{align}

\textbf{Misil (Interceptor):}
\begin{align}
    \mathbf{r}_M(0) &= \begin{bmatrix} 0 \\ -5\,000 \\ 2\,000 \end{bmatrix} \text{ m} \\
    \mathbf{v}_M(0) &= \begin{bmatrix} 350 \\ 80 \\ 90 \end{bmatrix} \text{ m/s} \\
    V_{cruise} &= 520 \text{ m/s} \approx 1\,872 \text{ km/h} \\
    V_{max} &= 650 \text{ m/s} \approx 2\,340 \text{ km/h} \\
    a_{max} &= 70 \text{ m/s}^2 \approx 7.1 \text{ G} \\
    N &= 4.5 \quad \text{(constante de navegación)}
\end{align}

\subsubsection{Estado Inicial del Sistema}

El vector de estado inicial completo es:
\begin{equation}
    \mathbf{S}(0) = \begin{bmatrix}
        20\,000 \\ 0 \\ 7\,000 \\ 250 \\ 0 \\ 0 \\ 0 \\ -5\,000 \\ 2\,000 \\ 350 \\ 80 \\ 90
    \end{bmatrix} \in \mathbb{R}^{12}
\end{equation}

\subsection{Cálculos Preliminares en $t=0$}

Evaluemos explícitamente todas las cantidades relevantes en el instante inicial.

\subsubsection{Geometría Relativa}

\textbf{Vector de posición relativa:}
\begin{align}
    \mathbf{r}_{rel}(0) &= \mathbf{r}_T(0) - \mathbf{r}_M(0) \\
    &= \begin{bmatrix} 20\,000 \\ 0 \\ 7\,000 \end{bmatrix} - \begin{bmatrix} 0 \\ -5\,000 \\ 2\,000 \end{bmatrix} \\
    &= \begin{bmatrix} 20\,000 \\ 5\,000 \\ 5\,000 \end{bmatrix} \text{ m}
\end{align}

\textbf{Distancia inicial:}
\begin{align}
    R(0) &= \|\mathbf{r}_{rel}(0)\| \\
    &= \sqrt{20\,000^2 + 5\,000^2 + 5\,000^2} \\
    &= \sqrt{400\,000\,000 + 25\,000\,000 + 25\,000\,000} \\
    &= \sqrt{450\,000\,000} \\
    &= 21\,213.2 \text{ m} \approx 21.21 \text{ km}
\end{align}

\textbf{Vector unitario de LOS:}
\begin{align}
    \hat{\mathbf{r}}_{LOS}(0) &= \frac{\mathbf{r}_{rel}(0)}{R(0)} \\
    &= \frac{1}{21\,213.2}\begin{bmatrix} 20\,000 \\ 5\,000 \\ 5\,000 \end{bmatrix} \\
    &= \begin{bmatrix} 0.9428 \\ 0.2357 \\ 0.2357 \end{bmatrix}
\end{align}

\subsubsection{Cinemática Relativa}

\textbf{Velocidad relativa:}
\begin{align}
    \mathbf{v}_{rel}(0) &= \mathbf{v}_T(0) - \mathbf{v}_M(0) \\
    &= \begin{bmatrix} 250 \\ 0 \\ 0 \end{bmatrix} - \begin{bmatrix} 350 \\ 80 \\ 90 \end{bmatrix} \\
    &= \begin{bmatrix} -100 \\ -80 \\ -90 \end{bmatrix} \text{ m/s}
\end{align}

\textbf{Producto punto para velocidad de cierre:}
\begin{align}
    \mathbf{r}_{rel}(0) \cdot \mathbf{v}_{rel}(0) &= 20\,000 \cdot (-100) + 5\,000 \cdot (-80) + 5\,000 \cdot (-90) \\
    &= -2\,000\,000 - 400\,000 - 450\,000 \\
    &= -2\,850\,000 \text{ m}^2\text{/s}
\end{align}

\textbf{Velocidad de cierre:}
\begin{align}
    V_c(0) &= -\frac{\mathbf{r}_{rel}(0) \cdot \mathbf{v}_{rel}(0)}{R(0)} \\
    &= -\frac{-2\,850\,000}{21\,213.2} \\
    &= 134.4 \text{ m/s}
\end{align}

Esto indica que el misil y el objetivo se están acercando a una velocidad de $134.4$ m/s $\approx 484$ km/h.

\subsubsection{Velocidad Angular de la LOS}

\textbf{Producto vectorial:}
\begin{align}
    \mathbf{r}_{rel}(0) \times \mathbf{v}_{rel}(0) &= \begin{vmatrix}
        \mathbf{i} & \mathbf{j} & \mathbf{k} \\
        20\,000 & 5\,000 & 5\,000 \\
        -100 & -80 & -90
    \end{vmatrix} \\
    &= \mathbf{i}(5\,000 \cdot (-90) - 5\,000 \cdot (-80)) \\
    &\quad - \mathbf{j}(20\,000 \cdot (-90) - 5\,000 \cdot (-100)) \\
    &\quad + \mathbf{k}(20\,000 \cdot (-80) - 5\,000 \cdot (-100)) \\
    &= \mathbf{i}(-450\,000 + 400\,000) - \mathbf{j}(-1\,800\,000 + 500\,000) \\
    &\quad + \mathbf{k}(-1\,600\,000 + 500\,000) \\
    &= \begin{bmatrix} -50\,000 \\ 1\,300\,000 \\ -1\,100\,000 \end{bmatrix} \text{ m}^2\text{/s}
\end{align}

\textbf{Velocidad angular:}
\begin{align}
    \boldsymbol{\Omega}(0) &= \frac{\mathbf{r}_{rel}(0) \times \mathbf{v}_{rel}(0)}{R(0)^2} \\
    &= \frac{1}{(21\,213.2)^2}\begin{bmatrix} -50\,000 \\ 1\,300\,000 \\ -1\,100\,000 \end{bmatrix} \\
    &= \frac{1}{450\,000\,000}\begin{bmatrix} -50\,000 \\ 1\,300\,000 \\ -1\,100\,000 \end{bmatrix} \\
    &= \begin{bmatrix} -0.000111 \\ 0.002889 \\ -0.002444 \end{bmatrix} \text{ rad/s}
\end{align}

\textbf{Magnitud de la velocidad angular:}
\begin{align}
    \|\boldsymbol{\Omega}(0)\| &= \sqrt{0.000111^2 + 0.002889^2 + 0.002444^2} \\
    &= 0.00379 \text{ rad/s} \approx 0.217^\circ\text{/s}
\end{align}

\subsection{Cálculo de la Aceleración de Comando PN en $t=0$}

\subsubsection{Producto Vectorial Triple}

Calculamos $\boldsymbol{\Omega}(0) \times \hat{\mathbf{r}}_{LOS}(0)$:
\begin{align}
    \boldsymbol{\Omega}(0) \times \hat{\mathbf{r}}_{LOS}(0) &= \begin{vmatrix}
        \mathbf{i} & \mathbf{j} & \mathbf{k} \\
        -0.000111 & 0.002889 & -0.002444 \\
        0.9428 & 0.2357 & 0.2357
    \end{vmatrix}
\end{align}

Evaluando cada componente:
\begin{align}
    \text{Componente } x &= 0.002889 \cdot 0.2357 - (-0.002444) \cdot 0.2357 \\
    &= 0.000681 + 0.000576 = 0.001257 \text{ 1/s} \\
    \text{Componente } y &= -((-0.000111) \cdot 0.2357 - (-0.002444) \cdot 0.9428) \\
    &= -(-0.0000262 + 0.002304) = -0.002278 \text{ 1/s} \\
    \text{Componente } z &= (-0.000111) \cdot 0.2357 - 0.002889 \cdot 0.9428 \\
    &= -0.0000262 - 0.002724 = -0.002750 \text{ 1/s}
\end{align}

Por tanto:
\begin{equation}
    \boldsymbol{\Omega}(0) \times \hat{\mathbf{r}}_{LOS}(0) = \begin{bmatrix} 0.001257 \\ -0.002278 \\ -0.002750 \end{bmatrix} \text{ 1/s}
\end{equation}

\subsubsection{Comando de Aceleración Pura PN}

Aplicando la ley PN con $N = 4.5$:
\begin{align}
    \mathbf{a}_{PN}(0) &= N \cdot V_c(0) \cdot (\boldsymbol{\Omega}(0) \times \hat{\mathbf{r}}_{LOS}(0)) \\
    &= 4.5 \cdot 134.4 \cdot \begin{bmatrix} 0.001257 \\ -0.002278 \\ -0.002750 \end{bmatrix} \\
    &= 604.8 \cdot \begin{bmatrix} 0.001257 \\ -0.002278 \\ -0.002750 \end{bmatrix} \\
    &= \begin{bmatrix} 0.76 \\ -1.38 \\ -1.66 \end{bmatrix} \text{ m/s}^2
\end{align}

Magnitud de la aceleración PN pura:
\begin{equation}
    \|\mathbf{a}_{PN}(0)\| = \sqrt{0.76^2 + 1.38^2 + 1.66^2} = 2.29 \text{ m/s}^2 \approx 0.23 \text{ G}
\end{equation}

\subsection{Evolución Temporal completa: Resultados de la Simulación}

\subsubsection{Métricas Finales}

Resultados obtenidos para el escenario **Espiral** (maniobra más representativa):

\begin{center}
\begin{tabular}{|l|r|l|p{6cm}|}
\hline
\textbf{Métrica} & \textbf{Valor} & \textbf{Unidad} & \textbf{Interpretación} \\
\hline
\hline
Tiempo de intercepción & 77.70 & s & Tiempo total de vuelo hasta impacto. El misil intercepta al objetivo en poco más de un minuto. \\
\hline
Distancia final (miss distance) & 41.8 & m & Distancia mínima alcanzada. $< 50$ m se considera impacto letal debido a la espoleta de proximidad. \\
\hline
Velocidad de cierre promedio & 273.3 & m/s & Velocidad relativa promedio de acercamiento $\approx 984$ km/h, indicando eficiencia en el enfoque. \\
\hline
Aceleración máxima (misil) & 70.1 & m/s\textsuperscript{2} & Aprox. 7.15 G. Dentro de los límites estructurales de misiles tácticos (20-30 G). \\
\hline
Distancia recorrida (misil) & 34.8 & km & Alcance efectivo del misil en esta trayectoria. Consistente con misiles SAM de medio alcance. \\
\hline
Número de iteraciones RK4 & 1554 & --- & Número de pasos temporales: $n = t_f / h = 77.70 / 0.05 = 1554$. \\
\hline
\end{tabular}
\end{center}

\subsection{Análisis de Convergencia Numérica}

\subsubsection{Dependencia del Paso de Tiempo}

Para verificar la convergencia del método RK4, se realizaron simulaciones con diferentes valores de $h$:

\begin{center}
\begin{tabular}{|c|c|c|c|}
\hline
$h$ (s) & $t_f$ (s) & $d_{miss}$ (m) & Error relativo \\
\hline
0.1 & 77.80 & 43.2 & Referencia \\
0.05 & 77.70 & 41.8 & 3.2\% \\
0.025 & 77.68 & 41.5 & 0.7\% \\
0.01 & 77.67 & 41.4 & 0.2\% \\
\hline
\end{tabular}
\end{center}

La variación mínima ($< 1$\%) entre $h = 0.05$ y $h = 0.01$ confirma la convergencia del método.

\subsubsection{Conservación de Invariantes}

Para un sistema conservativo (sin propulsión continua), la energía mecánica total debería conservarse. Definimos:
\begin{equation}
    E(t) = \frac{1}{2}m_M\|\mathbf{v}_M(t)\|^2 + \frac{1}{2}m_T\|\mathbf{v}_T(t)\|^2
\end{equation}

El error relativo de energía:
\begin{equation}
    \delta E = \frac{|E(t_f) - E(0)|}{E(0)} < 10^{-6}
\end{equation}

confirma la estabilidad numérica del método RK4.

%==============================================================================
\section{Conclusiones}
%==============================================================================

Este trabajo ha presentado un análisis matemático riguroso del sistema de ecuaciones diferenciales que gobierna la dinámica de intercepción de misiles en tres dimensiones. Los principales resultados son:

\begin{enumerate}
    \item El sistema se modela mediante 12 EDOs acopladas no lineales en $\mathbb{R}^{12}$
    \item La ley de Navegación Proporcional proporciona una estrategia eficiente que anula la velocidad angular de la línea de visión
    \item El método RK4 garantiza errores de truncamiento $\mathcal{O}(h^5)$, órdenes de magnitud menores que Euler
    \item El controlador híbrido combina PN, guiado predictivo y maniobras pre-programadas
    \item Los resultados numéricos demuestran intercepción exitosa con $d_{miss} < 50$ m
\end{enumerate}

%==============================================================================
\section{Referencias}
%==============================================================================

\begin{enumerate}
    \item Zarchan, P. (2013). \textit{Tactical and Strategic Missile Guidance} (6th ed.). AIAA Progress in Astronautics and Aeronautics.
    
    \item Ben-Asher, J. Z. (2017). \textit{Optimal Control Theory with Aerospace Applications}. AIAA Education Series.
    
    \item Siouris, G. M. (2004). \textit{Missile Guidance and Control Systems}. Springer.
    
    \item Press, W. H., et al. (2007). \textit{Numerical Recipes: The Art of Scientific Computing} (3rd ed.). Cambridge University Press.
    
    \item Shneydor, N. A. (1998). \textit{Missile Guidance and Pursuit: Kinematics, Dynamics and Control}. Horwood Publishing.
    
    \item Butcher, J. C. (2016). \textit{Numerical Methods for Ordinary Differential Equations} (3rd ed.). Wiley.
\end{enumerate}

\end{document}
